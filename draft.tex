%!TEX TS-program = xelatex
\documentclass[11pt]{article}

\usepackage[english]{babel}

\usepackage{amsmath,amssymb,amsfonts}
\usepackage[utf8]{inputenc}
\usepackage[T1]{fontenc}
\usepackage{stix2}
\usepackage[scaled]{helvet}
\usepackage[scaled]{inconsolata}

\usepackage{lastpage}

\usepackage{setspace}

\usepackage{ccicons}

\usepackage[hang,flushmargin]{footmisc}

\usepackage{geometry}

\setlength{\parindent}{0pt}
\setlength{\parskip}{6pt plus 2pt minus 1pt}

\usepackage{fancyhdr}
\renewcommand{\headrulewidth}{0pt}\providecommand{\tightlist}{%
  \setlength{\itemsep}{0pt}\setlength{\parskip}{0pt}}

\makeatletter
\newcounter{tableno}
\newenvironment{tablenos:no-prefix-table-caption}{
  \caption@ifcompatibility{}{
    \let\oldthetable\thetable
    \let\oldtheHtable\theHtable
    \renewcommand{\thetable}{tableno:\thetableno}
    \renewcommand{\theHtable}{tableno:\thetableno}
    \stepcounter{tableno}
    \captionsetup{labelformat=empty}
  }
}{
  \caption@ifcompatibility{}{
    \captionsetup{labelformat=default}
    \let\thetable\oldthetable
    \let\theHtable\oldtheHtable
    \addtocounter{table}{-1}
  }
}
\makeatother

\usepackage{array}
\newcommand{\PreserveBackslash}[1]{\let\temp=\\#1\let\\=\temp}
\let\PBS=\PreserveBackslash

\usepackage[breaklinks=true]{hyperref}
\hypersetup{colorlinks,%
citecolor=blue,%
filecolor=blue,%
linkcolor=blue,%
urlcolor=blue}
\usepackage{url}

\usepackage{caption}
\setcounter{secnumdepth}{0}
\usepackage{cleveref}

\usepackage{graphicx}
\makeatletter
\def\maxwidth{\ifdim\Gin@nat@width>\linewidth\linewidth
\else\Gin@nat@width\fi}
\makeatother
\let\Oldincludegraphics\includegraphics
\renewcommand{\includegraphics}[1]{\Oldincludegraphics[width=\maxwidth]{#1}}

\usepackage{longtable}
\usepackage{booktabs}

\usepackage{color}
\usepackage{fancyvrb}
\newcommand{\VerbBar}{|}
\newcommand{\VERB}{\Verb[commandchars=\\\{\}]}
\DefineVerbatimEnvironment{Highlighting}{Verbatim}{commandchars=\\\{\}}
% Add ',fontsize=\small' for more characters per line
\usepackage{framed}
\definecolor{shadecolor}{RGB}{248,248,248}
\newenvironment{Shaded}{\begin{snugshade}}{\end{snugshade}}
\newcommand{\KeywordTok}[1]{\textcolor[rgb]{0.13,0.29,0.53}{\textbf{#1}}}
\newcommand{\DataTypeTok}[1]{\textcolor[rgb]{0.13,0.29,0.53}{#1}}
\newcommand{\DecValTok}[1]{\textcolor[rgb]{0.00,0.00,0.81}{#1}}
\newcommand{\BaseNTok}[1]{\textcolor[rgb]{0.00,0.00,0.81}{#1}}
\newcommand{\FloatTok}[1]{\textcolor[rgb]{0.00,0.00,0.81}{#1}}
\newcommand{\ConstantTok}[1]{\textcolor[rgb]{0.00,0.00,0.00}{#1}}
\newcommand{\CharTok}[1]{\textcolor[rgb]{0.31,0.60,0.02}{#1}}
\newcommand{\SpecialCharTok}[1]{\textcolor[rgb]{0.00,0.00,0.00}{#1}}
\newcommand{\StringTok}[1]{\textcolor[rgb]{0.31,0.60,0.02}{#1}}
\newcommand{\VerbatimStringTok}[1]{\textcolor[rgb]{0.31,0.60,0.02}{#1}}
\newcommand{\SpecialStringTok}[1]{\textcolor[rgb]{0.31,0.60,0.02}{#1}}
\newcommand{\ImportTok}[1]{#1}
\newcommand{\CommentTok}[1]{\textcolor[rgb]{0.56,0.35,0.01}{\textit{#1}}}
\newcommand{\DocumentationTok}[1]{\textcolor[rgb]{0.56,0.35,0.01}{\textbf{\textit{#1}}}}
\newcommand{\AnnotationTok}[1]{\textcolor[rgb]{0.56,0.35,0.01}{\textbf{\textit{#1}}}}
\newcommand{\CommentVarTok}[1]{\textcolor[rgb]{0.56,0.35,0.01}{\textbf{\textit{#1}}}}
\newcommand{\OtherTok}[1]{\textcolor[rgb]{0.56,0.35,0.01}{#1}}
\newcommand{\FunctionTok}[1]{\textcolor[rgb]{0.00,0.00,0.00}{#1}}
\newcommand{\VariableTok}[1]{\textcolor[rgb]{0.00,0.00,0.00}{#1}}
\newcommand{\ControlFlowTok}[1]{\textcolor[rgb]{0.13,0.29,0.53}{\textbf{#1}}}
\newcommand{\OperatorTok}[1]{\textcolor[rgb]{0.81,0.36,0.00}{\textbf{#1}}}
\newcommand{\BuiltInTok}[1]{#1}
\newcommand{\ExtensionTok}[1]{#1}
\newcommand{\PreprocessorTok}[1]{\textcolor[rgb]{0.56,0.35,0.01}{\textit{#1}}}
\newcommand{\AttributeTok}[1]{\textcolor[rgb]{0.77,0.63,0.00}{#1}}
\newcommand{\RegionMarkerTok}[1]{#1}
\newcommand{\InformationTok}[1]{\textcolor[rgb]{0.56,0.35,0.01}{\textbf{\textit{#1}}}}
\newcommand{\WarningTok}[1]{\textcolor[rgb]{0.56,0.35,0.01}{\textbf{\textit{#1}}}}
\newcommand{\AlertTok}[1]{\textcolor[rgb]{0.94,0.16,0.16}{#1}}
\newcommand{\ErrorTok}[1]{\textcolor[rgb]{0.64,0.00,0.00}{\textbf{#1}}}
\newcommand{\NormalTok}[1]{#1}

\newlength{\cslhangindent}
\setlength{\cslhangindent}{1.5em}
\newlength{\csllabelwidth}
\setlength{\csllabelwidth}{3em}
\newenvironment{CSLReferences}[3] % #1 hanging-ident, #2 entry spacing
 {% don't indent paragraphs
  \setlength{\parindent}{0pt}
  % turn on hanging indent if param 1 is 1
  \ifodd #1 \everypar{\setlength{\hangindent}{\cslhangindent}}\ignorespaces\fi
  % set entry spacing
  \ifnum #2 > 0
  \setlength{\parskip}{#2\baselineskip}
  \fi
 }%
 {}
\usepackage{calc} % for \widthof, \maxof
\newcommand{\CSLBlock}[1]{#1\hfill\break}
\newcommand{\CSLLeftMargin}[1]{\parbox[t]{\maxof{\widthof{#1}}{\csllabelwidth}}{#1}}
\newcommand{\CSLRightInline}[1]{\parbox[t]{\linewidth}{#1}}
\newcommand{\CSLIndent}[1]{\hspace{\cslhangindent}#1}\geometry{verbose,letterpaper,tmargin=2.2cm,bmargin=2.2cm,lmargin=2.2cm,rmargin=2.2cm}

\usepackage{lineno}
\usepackage[nolists,noheads]{endfloat}

\pagestyle{plain}

\tolerance=1
\emergencystretch=\maxdimen
\hyphenpenalty=10000
\hbadness=10000

\doublespacing

\fancypagestyle{normal}
{
  \fancyhf{}
  \fancyfoot[R]{\footnotesize\sffamily\thepage\ of \pageref*{LastPage}}
}
\begin{document}
\raggedright
\thispagestyle{empty}
{\Large\bfseries\sffamily Thesis proposal}
\vskip 5em

%
\href{https://orcid.org/0000-0002-6506-6487}{Michael D.\,Catchen}%
%
\,\textsuperscript{1,2}

\textsuperscript{1}\,McGill University\quad \textsuperscript{2}\,Québec
Centre for Biodiversity Sciences


\textbf{Correspondance to:}\\
Michael D. Catchen --- \texttt{michael.catchen@mail.mcgill.ca}\\

\vfill
This work is released by its authors under a CC-BY 4.0 license\hfill\ccby\\
Last revision: \emph{\today}

\clearpage
\thispagestyle{empty}

\vfill
The proposal for my thesis, \emph{Simulation models for predictive
ecology}



\vfill

\clearpage
\linenumbers
\pagestyle{normal}

\hypertarget{introduction}{%
\section{Introduction}\label{introduction}}

\textbf{P1}

Within the last several hundred years, human activity has induced rapid
changes in Earth's atmosphere, oceans, and surface. Greenhouse gas
emissions have caused an increase the temperature of both Earth's
terrain and oceans, and both agricultural and urban development has
rapidly reshaped the Earth's land cover. These the bulk of this change
has occurred within the last several hundred years, a geological
instant, inducing a sudden shift in conditions to Earth's climate and
biosphere. As a result, predicting how ecosystems will change in the
future, \emph{ecological forecasting}, and then using these forecasts to
make decisions to mitigate the negative consequences of this change on
ecosystems, their functioning, and the services they provide to humans
has emerged as an imperative for ecology and environmental science
(Dietze 2017). However, robust prediction of ecological processes is, to
say the least, quite difficult (Beckage \emph{et al.} 2011; Petchey
\emph{et al.} 2015). This difficultly is compounded by a few factors,
the first being that sampling ecosystems is not easy. Ecological data is
often biased, noisey, and sparse in both space and time. The current
paucity of ecological data has resulted in much interest in developing
global systems for \emph{ecosystem monitoring} (Makiola \emph{et al.}
2020), which would systematize the collection of biodiversity data in
manner that makes detecting and predicting change more possible than at
the moment (Urban \emph{et al.} 2021).

\textbf{P2}

The second major challenge in ecological forecasting is that the
underlying dynamics of most ecological processes are unknown and instead
must be inferred from this (sparse) data. Much of the history of
quantitatively modeling ecosystems have been done in the language of
dynamical systems, describing how the value of an observable state of
the system, represented by a vector of numbers
\([x_1, x_2, \dots, x_n]^T = \vec{x}\) changes as over time, yielding
models in the form of differential equations in continuous-time
settings--\(\frac{dx}{dt} = f(x)\)-- or difference equations in
discrete-time settings--\(x_t = f(x_{t-1})\)--where
\(f:\mathbb{R}^n \to \mathbb{R}^n\) is an arbitrary function describing
how the system changes on a moment-to-moment basis (e.g.~in the context
of communities, \(f\) could be Lotka-Voltera, Holling-Type-III or
DeAngelis-Beddington functional response). The initial success of these
forms of models can be traced back to the larger program of ontological
reductionism, which became the default approach to modeling in the
sciences after its early success in physics, which, by the time ecology
was becoming a quantitative science (sometime in the 20th century,
depending on who you ask), became the foundation for early quantitative
models in ecology.

\textbf{P3}

However, we run into many problems when aiming to apply this type of
model to empirical data in ecology. Ecosystems are perhaps the
quintessential example of system that cannot be understood by iterative
reduction of its components into constituent parts---ecological
phenomena are emergent are the product of different mechanisms operating
a different spatial, temporal, and organizational scales (Levin 1992).
Further, the form of this functional response in real systems is
effectively unknown, and some forms are inherently more ``forecastable''
than others (Beckage \emph{et al.} 2011; Chen \emph{et al.} 2019;
Pennekamp \emph{et al.} 2019). Further this analytical approach to
modeling explicitly ignores known realities: ecological dynamics not
deterministic, many analytic models in ecology assume long-run
equilibrium. Finally, perhaps the biggest challenge in using these
models to describe ecological processes is ecosystems vary across more
variables than the tools of analytic models are suited for. As the
number of variables in an analytic model increases, so does the ability
of the scientist to discern clear relationships between them given a
fixed amount of data, the so-called ``curse'' of dimensionality.

\textbf{P4}

But these problems are not solely unique to ecology. The term
\emph{ecological forecasting} implicitly creates an analogy with weather
forecasting. Although it has become a trite joke to complain about the
weather forecast being wrong, over the least 50 years the field of
numerical weather prediction (NWP) has dramatically improved out ability
to predict weather across the board (Bauer \emph{et al.} 2015). The
success of NWP, and the Earth observations systems that support it (Hill
\emph{et al.} 2004), should serve as a template for development of a
system for monitoring Earth's biodiversity. Much like ecology, NWP is
faced with high-dimensional systems that are governed by different
mechanisms at different scales. The success of NWP is that, rather than,
say, attempt to forecast the weather in Quebec by applying Navier-Stokes
to entire province, to instead use simulation models which describe
known mechanisms at different scales, and use the availability to
increasing computational power to directly simulate many batches of
dynamics which directly incorporate stochasticity and uncertainty in
parameter estimates via random number generation.

\textbf{P6}

But forecasting is only half the story. Marx's most well known quote
that ``philosophers have hitherto only interpreted the world in various
ways; the point is to change it.'' Indeed, once we have a forecast about
how an ecosystem will change in the future, what if this forecast
predicts a critical ecosystem service will deteriorate? We are still
left with the question, what do we in the time being to mitigate the
negative consequences a forecast predicts? In this framing, mitigating
the consequences of anthropogenic change on ecosystems becomes an
optimization problem: given a forecast of the probability. We have some
goal state for the future, and some estimate of what the state of the
world will be given a set of actions. Frame optimization problem
mathematically an introduce concept of solution-space and constraint.

\begin{figure}
\hypertarget{fig:thesis}{%
\centering
\includegraphics{./figures/thesisconcept.png}
\caption{thesis concept}\label{fig:thesis}
}
\end{figure}

\textbf{P7}

This dissertation aims to formalize a framework for ecosystem monitoring
and forecasting (fig.~\ref{fig:thesis}, left), and each chapter address
some aspect of this pipeline to data from a monitoring network to
forecasts to mitigation strategy (fig.~\ref{fig:thesis}, right).

\hypertarget{chapter-one-forecasting-the-spatial-uncoupling-of-a-plant-pollinator-network}{%
\section{Chapter One: Forecasting the spatial uncoupling of a
plant-pollinator
network}\label{chapter-one-forecasting-the-spatial-uncoupling-of-a-plant-pollinator-network}}

Plants and pollinators form interaction networks, called the
``architecture of biodiversity'' (\textbf{Jordano2007?}).

The stability, function, and persistance of ecosystems relies on the
structure of these interactions. Antropogenic change threatens to
unravel these networks. Two aspects to this change: spatial and
temporal. Spatially, range shifts along elevational gradient, and
temporall, phenological shifts.

The issue is that we don't really know what interactions are like now.
So not only do we need to predict with data that is spatially and
temporally sparse and likely to contain many interaction
``false-negatives'' (Strydom \emph{et al.} 2021).

This chapter uses several years of data on bee-flower phenology and
interactions, combined with spatial records of species occurrence via
GBIF, to forecast how much overlap there will be between
plants/pollinators in space/time.

In stages, (1) take data from multiple sites to predict a spatial
metaweb of \emph{Bombus}-flower interactions across Colorado. (2)
Predict how these spatial distributions will change under CMIP6. and (3)
quantify the lack of overlap between species for which there is a
predicted

\begin{figure}
\centering
\includegraphics{./figures/ch1.png}
\caption{chapter one concept fig}
\end{figure}

\hypertarget{data}{%
\subsection{Data}\label{data}}

The data for this chapter is derived from multiple souces and can be
split into three categories. (1) Field data from three different
locations across Colorado. All field sites have multiple plots across an
elevational gradient.

System description: lots of data on \emph{Bombus} (bumblebees) and
wildflowers. Three different sites, (7/7/3) years each, each covering an
elevational gradient.

\hypertarget{methods}{%
\subsection{Methods}\label{methods}}

Split the process into parts.

\begin{enumerate}
\def\labelenumi{\arabic{enumi})}
\tightlist
\item
  Building an interaction prediction model. 2) Make it spatial based on
  distributions. 3) Forecast distributions based on CMIP6.
\end{enumerate}

\hypertarget{preliminary-results}{%
\subsection{Preliminary Results}\label{preliminary-results}}

\begin{enumerate}
\def\labelenumi{\arabic{enumi})}
\tightlist
\item
  we got a tree
\end{enumerate}

Transition to next chapter by discussing uncertainty in interaction
prediction across space.

\hypertarget{chapter-two-optimizing-spatial-sampling-of-species-interactions}{%
\section{Chapter Two: Optimizing spatial sampling of species
interactions}\label{chapter-two-optimizing-spatial-sampling-of-species-interactions}}

There are false-negatives in interation data. Co-occurrence is not the
same thing as interaction (\textbf{cite?}), but often is used as a
proxy.

This chapter unravels the relationship between a given species relative
abundance and the sampling effort needed to adequately understand this
species distribution and interactions.

There is more than one way to observe a false-negative.

\begin{figure}
\centering
\includegraphics{./figures/ch2.png}
\caption{taxonomy of false negatives}
\end{figure}

It begins with a conceptual framework for understanding the difference
in false-negatives in occurrence, co-occurrence, and interactions
(fig.~3). We use a null model of the relative-abundance distribution
(Hubbell 2001) to simulate realized false-negatives as a function of
varying sampling effort.

This also goes on to includes testing some assumptions of the model with
empirical data fig.~\ref{fig:posassoc}. which indicate our neutral
model, if anything, underestimates the probability of false-negatives
due to positive correlations in co-occurrence in two spatially
replicated networks (Thompson \& Townsend 2000; Hadfield \emph{et al.}
2014)---further I'm planning to add the field data from chapter one into
this anlysis once complete.

\begin{figure}
\hypertarget{fig:posassoc}{%
\centering
\includegraphics{./figures/positiveassociations.png}
\caption{f}\label{fig:posassoc}
}
\end{figure}

new addition: - simulate species distribution and efficacy of detection
given a set of observation points where the dist from observation site
decays. optimize set of repeated sampling locations L for a \emph{known}
distribution D. address SDM not being the territory

\hypertarget{results}{%
\subsection{Results}\label{results}}

\begin{itemize}
\tightlist
\item
  nonrandom association figure sampling effort under neutral model
\end{itemize}

\hypertarget{chapter-three-optimizing-corridor-placement-against-ecological-dynamics}{%
\section{Chapter Three: Optimizing corridor placement against ecological
dynamics}\label{chapter-three-optimizing-corridor-placement-against-ecological-dynamics}}

Promoting landscape connectivity is important to mitigate the effects of
land-use change on Earth's biodiversity. However, the practical
realities of conservation mean that there is a limitation on how much we
can modify landscapes in order to do this. So what is the best place to
put a corridor given a constraint on how much surface-area you can
change in a landscape? This is the question this chapter seeks to
answer. Models for proposing corridor locations have been developed, but
are limited in that are not developed around promoting some element of
ecosystem function, but instead by trying to find the path of least
resistance given a resistance surface (Peterman 2018).

This chapter proposes a general algorithm for optimizing corridor
placement based on a measurement of ecosystem functioning derived from
simulations run on a proposed landscape modification. We propose various
landscape modifications which alter the cover of a landscape,
represented as a raster (fig.~6, left). We then compute a new resistance
surface based on the proposed landscape modification, and based on the
values of resistance to dispersal between each location we simulate
spatially-explicit metapopulation dynamics model (Hanski \& Ovaskainen
2000; Ovaskainen \emph{et al.} 2002) to estimate a distribution of time
until extinction for each landscape modification (fig.~6, right).

\hypertarget{methods-1}{%
\subsection{Methods}\label{methods-1}}

\begin{itemize}
\tightlist
\item
  land cover -\textgreater{} resistance -\textgreater{} extinction time
  simulated annealing to
\item
  optimize landscape optimization
\end{itemize}

\hypertarget{ch4-a-software-note-on-the-resulting-packages.}{%
\section{CH4 a software note on the resulting
packages.}\label{ch4-a-software-note-on-the-resulting-packages.}}

(MetacommunityDynamics.jl: a virtual laboratory for community ecology):
a collection of modules in the Julia language for different aspects of
metacommunity ecology, including most of the code used for the preceding
chapters.

\begin{figure}
\centering
\includegraphics{./figures/ch4.png}
\caption{todo}
\end{figure}

\hypertarget{conclusion}{%
\section{Conclusion}\label{conclusion}}

\hypertarget{appendix}{%
\section{Appendix}\label{appendix}}

\begin{figure}
\centering
\includegraphics{./figures/trees.png}
\caption{trees}
\end{figure}

\hypertarget{references}{%
\section*{References}\label{references}}
\addcontentsline{toc}{section}{References}

\hypertarget{refs}{}
\begin{CSLReferences}{1}{0}
\leavevmode\hypertarget{ref-Bauer2015QuiRev}{}%
Bauer, P., Thorpe, A. \& Brunet, G. (2015). The quiet revolution of
numerical weather prediction. \emph{Nature}, 525, 47--56.

\leavevmode\hypertarget{ref-Beckage2011LimPre}{}%
Beckage, B., Gross, L.J. \& Kauffman, S. (2011). The limits to
prediction in ecological systems. \emph{Ecosphere}, 2, art125.

\leavevmode\hypertarget{ref-Chen2019RevCom}{}%
Chen, Y., Angulo, M.T. \& Liu, Y.-Y. (2019). Revealing Complex
Ecological Dynamics via Symbolic Regression. \emph{BioEssays}, 41,
1900069.

\leavevmode\hypertarget{ref-Dietze2017PreEco}{}%
Dietze, M.C. (2017). Prediction in ecology: A first-principles
framework. \emph{Ecological Applications}, 27, 2048--2060.

\leavevmode\hypertarget{ref-Hadfield2014TalTwo}{}%
Hadfield, J.D., Krasnov, B.R., Poulin, R. \& Nakagawa, S. (2014). A Tale
of Two Phylogenies: Comparative Analyses of Ecological Interactions.
\emph{The American Naturalist}, 183, 174--187.

\leavevmode\hypertarget{ref-Hanski2000MetCap}{}%
Hanski, I. \& Ovaskainen, O. (2000). The metapopulation capacity of a
fragmented landscape. \emph{Nature}, 404, 755--758.

\leavevmode\hypertarget{ref-Hill2004ArcEar}{}%
Hill, C., DeLuca, C., Balaji, Suarez, M. \& Da Silva, A. (2004). The
architecture of the Earth System Modeling Framework. \emph{Computing in
Science Engineering}, 6, 18--28.

\leavevmode\hypertarget{ref-Hubbell2001UniNeu}{}%
Hubbell, S.P. (2001). \emph{The unified neutral theory of biodiversity
and biogeography}. Monographs in population biology. Princeton
University Press, Princeton.

\leavevmode\hypertarget{ref-Levin1992ProPat}{}%
Levin, S.A. (1992). The Problem of Pattern and Scale in Ecology: The
Robert H. MacArthur Award Lecture. \emph{Ecology}, 73, 1943--1967.

\leavevmode\hypertarget{ref-Makiola2020KeyQue}{}%
Makiola, A., Compson, Z.G., Baird, D.J., Barnes, M.A., Boerlijst, S.P.,
Bouchez, A., \emph{et al.} (2020). Key Questions for Next-Generation
Biomonitoring. \emph{Frontiers in Environmental Science}, 7.

\leavevmode\hypertarget{ref-Ovaskainen2002MetMod}{}%
Ovaskainen, O., Sato, K., Bascompte, J. \& Hanski, I. (2002).
Metapopulation Models for Extinction Threshold in Spatially Correlated
Landscapes. \emph{Journal of Theoretical Biology}, 215, 95--108.

\leavevmode\hypertarget{ref-Ovaskainen2002MetMod}{}%
Ovaskainen, O., Sato, K., Bascompte, J. \& Hanski, I. (2002).
Metapopulation Models for Extinction Threshold in Spatially Correlated
Landscapes. \emph{Journal of Theoretical Biology}, 215, 95--108.

\leavevmode\hypertarget{ref-Pennekamp2019IntPre}{}%
Pennekamp, F., Iles, A.C., Garland, J., Brennan, G., Brose, U., Gaedke,
U., \emph{et al.} (2019). The intrinsic predictability of ecological
time series and its potential to guide forecasting. \emph{Ecological
Monographs}, 89, e01359.

\leavevmode\hypertarget{ref-Petchey2015EcoFor}{}%
Petchey, O.L., Pontarp, M., Massie, T.M., Kéfi, S., Ozgul, A.,
Weilenmann, M., \emph{et al.} (2015). The ecological forecast horizon,
and examples of its uses and determinants. \emph{Ecology Letters}, 18,
597--611.

\leavevmode\hypertarget{ref-Peterman2018ResRP}{}%
Peterman, W.E. (2018). ResistanceGA: An R package for the optimization
of resistance surfaces using genetic algorithms. \emph{Methods in
Ecology and Evolution}, 9, 1638--1647.

\leavevmode\hypertarget{ref-Strydom2021RoaPre}{}%
Strydom, T., Catchen, M.D., Banville, F., Caron, D., Dansereau, G.,
Desjardins-Proulx, P., \emph{et al.} (2021). \emph{A Roadmap Toward
Predicting Species Interaction Networks (Across Space and Time)}
(Preprint). EcoEvoRxiv.

\leavevmode\hypertarget{ref-Thompson2000ResSol}{}%
Thompson, R.M. \& Townsend, C.R. (2000). Is resolution the solution?:
The effect of taxonomic resolution on the calculated properties of three
stream food webs. \emph{Freshwater Biology}, 44, 413--422.

\leavevmode\hypertarget{ref-Urban2021CodLif}{}%
Urban, M.C., Travis, J.M.J., Zurell, D., Thompson, P.L., Synes, N.W.,
Scarpa, A., \emph{et al.} (2021). Coding for Life: Designing a Platform
for Projecting and Protecting Global Biodiversity. \emph{BioScience}.

\end{CSLReferences}

\end{document}
